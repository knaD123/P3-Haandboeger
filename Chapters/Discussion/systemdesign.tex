\section{System design \& iterations}\label{sec:dissystemdes}

In \cref{sec:systemdesign} the component design was presented and was implemented as shown in \cref{sec:codeexamples}.
Though despite the layout of the written report the reality is not as linear as simply designing the system and then implementing it afterwards.
This is conveyed through the several iteration the system went through described in \cref{sec:workflow}.

The architecture design of the system and the implementation of it happened synchronously throughout the iterations as the system requirements and the usability of the system was discovered through these iterations.
Because of this the design of the system was discovered through several iterations whereas the final iteration of the component design is presented in \cref{sec:systemdesign} and the interface design as seen in \cref{sec:Mock} and \ref{sec:2Mock}.
Discussion of the iterative model has occured in \cref{sec:iterativModel}.
This section seeks to add additional advantages and disadvantages that the project group identified throughout the process.

One of the advantages are that the architecture design and the implementation evolved at the same time, and there exist no disparrancy between either one.
As a result the implementation closely reflects the system design as these were written synchronously.

As written in \cref{sec:dissystemdef} all of the system requirements were fulfilled and the implementation did not suffer in that sense.
Though the problem occured during the development of the system's interface.
As illustrated in \cref{tab:utest2} in \cref{fourthtest} there were problems with the interface being not intuitive in relation with interacting with the system. 

One of the main disadvantages or perhaps failings with the development process were that the system interface designs were rushed and occured at the same time or after the interface implementation.
This has led to an interface that is not always intuitive and that the users had trouble interacting with.

One of the disadvantages is that the system and its features were at time implemented without the existense of the design.
Instead the system were at times designed based on the implementation.
This occured even though early drafts of the interface designs were prototyped during the first iteration as written in \cref{sec:Iteration1}.
Though these were only rough draft consisting of the rough layout of the interface and did not include finer details.
Finer details could for example include exactly what happened when a user interacted with the system, how the navigation worked, or as detailed as what happened when a specific button was pressedas detailed as what happened when a specific button was pressed

As written in \cref{sec:2Iteration-timeline} the designing of the interface began in the second iteration.
The design of the interface continued well into the third iteration.
Though these interface designs took place during the second and third iterations, they were completed using the waterfall method.
A description of the waterfall method can be found in \cref{sec:WaterfallModel}.

For the project this means whenever a new interface mockup was made, though not considered done, they were not iterated upon.
For example the interface for the login page and the index page of the system would be made and would not necessarily be updated or touched upon again.
A criticism can be made of this, as all other aspects of the system and its analysis went through iterations, but the interace designs and mockups did not.

A direct result of this is that the current interface implementation is not completely thought out and is not designed to be as intuitive as it could have been.

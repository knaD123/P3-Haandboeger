\section{Browser support}
In development JavaScript was used for site functionality, more specifically JavaScript was used to make notifications in the program but although JavaScript allowed the site to do so, not all browsers add support for new JavaScript and CSS features as soon as the new specs are released.
Although the latest version of Internet Explorer was released several years ago, it was said that the users of this system have little IT experience in \cref{factor}, which means it is very possible that they are actually using this browser.
The software was never tested using the browser, making it unclear if any nonsupported features are used in the frontend.
It could also have been solved by generating notifications using other tools or simply sending notification when opening the system, since this is what you are looking for notifications.
% Rasmus: Det lyder lidt som om vi siger at IE ikke har support for javascript
% Rasmus: Måske burde vi nævne vi ikke har testet det i IE, eller bare teste det i IE.
% Henrik: Jeg har prøvet at omformulere det lidt - jeg er dog lidt usikker paa om jeg maa nævne Babel her, naar det ikke er nævnt før? :-)

\section{Ajax}

Furhermore, in the system state right now, all text in a completed form will disappear if parts of the form are not validated.

This could be solved using Ajax, which is a development technique that can make a web application send and retrieve data from a server in the background, without interfering with and existing page.

It can store data in a completed form if errors occur on the page, this means it can update parts of a page without reloading a full page.
However, the problem was only discovered in the last iteration, so a lack of time was cause for it not to be resolved.

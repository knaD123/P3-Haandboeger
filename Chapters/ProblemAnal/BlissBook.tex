\subsection{BlissBook}
%\begin{figure}[h]
%	\includegraphics[width=1\textwidth]{billeder/BlissBooks.png}
%	\caption{BlissBook landing page}
%\end{figure}
% Anja: Hvorfor er det relevant at vise et billede af hjemmesiden? Jeg vil gerne cutte den fuldstændig, da det er fluff.
%Astrid: Alternativt skulle man inkludere screenshots fra hvordan deres håndbog fungerer i stedet
% husk at referere til figuren i teksten
BlissBook is a \textit{Software as a Service} (SaaS) employee handbook system, developed by the company of the same name starting in 2013\cite{BlissbookInfo}.
This chapter is mostly based upon information from their website\cite{BlissbookContents}.
% Anja: Cut url'en og sæt det i litteraturlisten. Det er ikke relevant at nævne, at vi har informationen fra en hjemmeside direkte i teksten.

BlissBook is based on a rich text editor which supports collabrative editing, and has features such as access control, digital signatures to keep track of which employees have read the documents, and sending notifications to employees when new versions are released.

As they are a SaaS platform, all the content and software is actually hosted on their servers.
In SU, this would be a local presentation client-server pattern. % SU? hvis det er system development er det noget man burde nævne her? 
On their server, they guarantee a 99.9\% uptime, and exports of all data on the platform, even after a canceled subscription.
How long they keep the data is not specified.

On the security front, the BlissBook website does not mention any certificatations on their software, besides they offer AES-256 encryption.\cite{BlissbookSecurity}
They do not freely offer up any sort of self-hosting solution, which security-conscious corporations may be very interested in.
But as Ipsen does not have any specific security concern besides data integrity, the SaaS structure is in the end of an advantage.

One of the greatest disadvantages to this system is that all the files in the system need to be written in BlissBook's own online editor.
This would present a great difficulty, as all existing documentation which Ipsen relies on is written mostly in word documents, but also spreadsheets, pictures, and others.
% det ved word document og andre filtyper  ligger det ikke senere i rapporten i bruger analyse?
% Henrik: Måske bare \cref det senere?

Additionally, the solution does not seem to be fitted to the use case of standards compliance, but rather handbooks for human resources use cases.
While they do have a system for external handbook consultants, they are again focusing on HR consultants.\cite{BlissbookHandbook}

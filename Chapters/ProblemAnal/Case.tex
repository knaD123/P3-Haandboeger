\section{Case description} \label{case}

This project will be based on the case presented by Ipsen.
Ipsen works with handbook documents where she is a quality manager for several firms.
The handbooks consists of several documents that each describe procedures or guidelines for different aspects of a firm and its workings.
There exists several versions of each document where only the newest version is active, and all of the previous versions are archived for approximately three to five years.
A new version of a document must be written at least once a year, and before it is active it must be approved by one to two persons.
Each document has a title with an ID that can only be associated with that specific title. 
When a title has been paired with an ID, the ID can no longer be associated with any other titles, even if the current title should be deleted.

Ipsen works together with secretaries who works in-house at the different firms.
The secretaries has the responsibility of mangaging the handbook documents and the everyday workers who works at the firm.
Whenever a new document has been updated it is imperative that the workers read and understand the documents relevant to them.
Not all documents are relevant for each worker, as the daily responsibilities of the workers can vary.

The problem with the current work procedure is that all the documents are stored manually in folders on a computer and Ipsen must archive older documents manually to separate archiving folders.
This manual work method takes time and it is complicated to keep track of the newest documents and their approval stages.

Ipsen requires a system that can manage the documents and their versions.
The system should furthermore be used to manage that the everyday workers have read the relevant documents. 

The next section describes which relevant standard or requirement that must be taken to account regarding handbook management. 

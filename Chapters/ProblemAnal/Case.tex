In the following chapter the problem and its analysis will be presented and explored. 
To gather information about the problem of the case an interview has been held with the stakeholder for the project, Pia Ipsen, who will be called Ipsen in the following report unless stated otherwise.
The knowledge from the interview will be analysed with the PACT-analysis in section xx.
Furthermore existing solutions will be presented and discussed in relation to their pros and cons.
The entire analysis will lead up to the problem statement which will guide the rest of the project.

\section{Case description} \label{case}

This project will be based on the case presented by Ipsen.
Ipsen works with handbook documents where she's a consultant or acting quality manager for several firms.
The handbooks consists of several documents that each describe procedures or guidelines for different aspect of a firm and its workings.
There exists several versions of each document where only the newest version is active, and all of the previous versions are archived for approximately 3-5 years.
A new version of a document must be written at least once a year, and before it is active it must be approved by 1-2 persons.
Furthermore, depending on the firm, new version of a document 

The problem with the current work procedure is all the documents are stored manually in folders on a computer and Ipsen must archive older documents manually to a separate archiving folder. This manual work method takes time and it is complicated to keep track of the newest documents and their approval stages.

\section{Case description} \label{sec:CaseDescription}

This project will be based on the case presented by Ipsen. 
Ipsen is a senior consultant for the firm QMS consult where they among other things help companies certified management systems require document management in the form of handbooks.
In her work she helps these companies with their handbooks and in a few cases acts as the companys quality manager.
All information in this section is based off of interviews with her.

The handbooks consist of several documents that each describe procedures or guidelines for different aspects of a firm and its workings.
There exists several versions of each document where only the newest version is active, and all of the previous versions are required to be archived for at least three to five years. All of these documents must adhere to a set of rules, notably the following:

\begin{itemize}
\item A document must always be up-to-date.
This generally means an at least yearly review of the document.
\item When a new version is written it must be approved by usually one or two people before it is active.
\item Each document has a title with an ID that can only be associated with that specific title. 
When a title has been paired with an ID, the ID can no longer be associated with any other titles, even if the current title should be deleted.
\end{itemize}

While the documents are only required to be archived for three to five years in reality previous versions are seldomly ever deleted due to the hassle of selecting files that are no longer needed and the relatively small space required to store them.

Ipsen works together with the firms to make sure that their handbooks are in good shape for audit.
%her kunne man indsætte information om standarder???
The secretaries have the responsibility of mangaging the handbook documents and the everyday workers who work at the firm.
Whenever a new document has been updated it is imperative that the workers read and understand the documents relevant to them and do so immediately.
Not all documents are relevant for each worker, as the daily responsibilities of the workers can vary.

The problem with the current work procedure is that all the documents are stored in folders on a computer and Ipsen must manually archive older documents by moving them to separate archiving folders.
This manual work method takes time and has a high risk of mistakes: 
It can be complicated to keep track of the newest documents and their approval stages. 

Ipsen requires a system that can manage the documents and their versions.
The system should furthermore be used to ensure that the everyday workers have read the relevant documents.

A more in depth analysis of Ipsens case will be explored in the FACTOR-analysis in \cref{sec:factorcriteria} and rich picture analysis is \cref{sec:richpictures}.

In relation to the handbook that the firms must have and upkeep, there exist specific standards for how to maintain these handbooks.
These standards will be elaborated in the next section.

%?The next section describes which relevant standard or requirement that must be taken to account regarding handbook management. 
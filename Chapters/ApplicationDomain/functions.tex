\section{Functions} \label{sec:functions}
When creating a system it is vital to get an overview of the functions and to determine the complexity thereof.
This is done by an nalysis of functions, in which the first step is to identify all functions.
The result is a list of functions with a related complexity and type.
This section is based on Mathiassen, Munk-Madsen, Nielsen and Stage \citep[ch.~7]{Rod-Aalborg}.

To commence the analysis, a look at the system definition and use cases is necessary, see \cref{sec:SystemDefinition,sec:usecases}, as it creates a template of functions that are ready to be analysed.
The next step is then to determine what type each function belong to, this means determining if it is an \textit{update, signal, read} or \textit{compute function}.
A description of each of these function types can be found in the following item list, \citep[p.~140]{Rod-Aalborg}:

\begin{itemize}
	\item
	''\textit{Update} functions are  activated by a problem-domain event and result in a change in the model's state.
	\item
	''\textit{Signal} functions are activated by a change in the model's state and result in a reaction in the context; \ldots
	\item
	''\textit{Read} functions are activated by a need for information in an actor's work task and result in the system displaying relevant parts of the model.
	\item
	''\textit{Compute} functions are activated by a need for information in an actor's work task and consist of a computation involving information provided by the actor or the model; \ldots''
\end{itemize}

The functions in this project, shown in the function table below, are identical to the qualified events in \cref{sec:Events}.
This is because update functions should reflect the system's classes and events.

\begin{table}[H]
\centering
\begin{tabular}{lll}
	\hline
	Function						& Complexity & Type    \\
	\hline
	Add document					& Simple     & Update  \\
	Archive document				& Simple     & Update  \\
	Delete document					& Simple     & Update  \\
	Add version						& Medium	 & Update  \\
	Approve version					& Simple     & Update  \\
	Mark version as read			& Medium     & Update  \\
	Archive version					& Simple     & Update  \\
	Add user						& Simple     & Update  \\
	Update user						& Simple     & Update  \\
	Delete user						& Simple     & Update  \\
	Request approval				& Medium     & Update  \\
	Delete approval					& Simple     & Update  \\
	Add department					& Simple     & Update  \\
	Delete department				& Simple	 & Update  \\
	Add user to department			& Simple     & Update  \\
	Remove user from department		& Simple     & Update  \\
	Add document to department		& Simple     & Update  \\
	Remove document from department & Simple     & Update  \\
	Notify user						& Simple	 & Signal  \\
	\hline
\end{tabular}
\caption{Function list}
\end{table}

The most present type of function in this system is the update type.
The reason for this is in the nature of the system, as it primarily lies in management of users and documents.
This also means that the complexity is mostly simple as the margin for error is rather small in these cases.
This is because of the concreteness of the functions, as each has minimal amount of uncertainty when used.

The "Notify user" is of the type signal, a function type that responds to changes inside the system.

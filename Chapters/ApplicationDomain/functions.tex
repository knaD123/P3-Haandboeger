\section{Functions}
When creating a system it is vital to get an overview of the functions and to determine the complexity of these functions.
This is done by analysis of functions, in which the first step is to identify all functions.
The result is a list of functions with a related complexity and type.
This is taken from chapter $7$ in the OOA\&D.\citep[ch.~7]{Rod-Aalborg}
%Anja: Slet den sidste sætning.

%Taniya: some word is missing in the first sentence 
To begin the analysis a look at the system definition and use cases is necessary, see \cref{sec:systemdefinition} and \cref{sec:usecases}, as it creates a template of functions that are ready to be analysed.
The next step is then to determine what type each function belongs to, this means determining if it is an \textit{update, signal, read or compute function}.
The most present type of function present are of the update type.
These are described in the OOA\&D as "Update functions are activated by a problem-domain event and result in a change in the model's state."\citep[p.~140]{Rod-Aalborg}

%Taniya: Måske tilføje en kort beskrivelse af update, signal, read or compute

The reason in this is in the nature of the system, as it primarily lies in management of users and documents.
This also means that the complexity is mostly simple as the margin for error is rather small in these cases. 

Only exceptions are "Manage documents", where the complexity is in the validation where overlapping of information is not allowed.
Furthermore "Track differences between document versions" can be difficult as different file types can be used by the users.

The "Notify user" is of the type signal, a function type that responds to changes inside the system.

"Track differences between document versions" is of the type read as it retrieves information from each document and presents the differences to the user.

This is the resulting table of functions:

\begin{table}[H]
\centering
\begin{tabular}{lll}
Function                                    & Complexity & Type    \\
Manage documents                            & Medium     & Update  \\
Manage users                                & Simple     & Update  \\
Manage departments                          & Simple     & Update  \\
Manage suppliers                            & Simple     & Update  \\
Approve new suppliers                       & Simple     & Update  \\
Approve new documents                       & Simple     & Update  \\
Update TOC                                  & Simple     & Update  \\
Track differences between~document versions & Medium     & Read    \\
Notify user                                 & Simple     & Signal  \\
Read status                                 & Simple     & Update 
\end{tabular}
\caption{Function list}
\end{table}


% Taniya: mener manage documents ikke rigtig er en funktion i sig selv men  der findes flere funktioner som til sammen gør det muligt at manage documents?

%Anja: Manage users, Manage documents osv. er ikke funktioner, men use cases. Vi har lavet en tabel over funktioner i vores google drive mappe. Tabellen ovenfor skal lige have en overhaul og generelt mere beskrivelse ind af enten alle funktioner eller blot nøglefunktioner.

% Taniya: Synes at table format skal være den samme som dem i Class and Events afsnit
% Henrik: Synes dette kapitel virker meget "tyndt"?
% Henrik: Der mangler en "scope of application domain"/"summary" for hele application kapitlet :-)

% Anja: Generelt for afsnittet mangler der nok mere metatekst - f.eks. åbning og slutning
% Henrik: Anja please fix merge conflicts før du pusher :-P

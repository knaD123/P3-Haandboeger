
In this chapter the application domain will be analyzed.
Where the problem domain, \cref{ch:ProblemDomain}, focused on; what information the system should process, the application domain focuses on; how the user will use the system in practice.
This analysis will start with a PACT analysis where the context of the use of the system will be explored which delves into the people, their activities, contexts and the technologies used.
Afterwards use cases of the system will be explored followed by functions that need to support these use cases.

\section{PACT analysis}\label{sec:PACT}
The theory of this section is based upon Benyon \cite{Benyon} and the analysis made during the project unless anything else is stated.

\todo[inline]{Andreas: Udskriv machine-centered}
%Anna: Tjek op på om den todo er lavet og evt fjern den hvis det er i orden.
When developing an interactive system it is important to design it with the users in mind as the system is being developed for a specific purpose and context.
This is also know as human-centred design.
On the other hand is the machine-centred design where the users are seen as vague, disorganized and have to fit the needs of the machine. 
%Anna: Fjernede denne linje, synes ikke den passede ind, og synes det ovenfor fint beskrev det
%In such way the data might have a special format or users might have to act in a certain pattern.
To design human-centred the developers must explore the users who will eventually use the system and for what purposes they are going to utilize it.

To help understand and reflect upon the users and their relation to an interactive system, the framework \textit{PACT} is used.
PACT is an acronym for \textit{People, Activities, Context}, and \textit{Technology}.
%Anna: med det tidligere afsnit virker denne linje ligegyldig
%This analysis framework is used to analyze the users, such that developers are able to design the system human-centered rather than machine-centered.

Depending on how well developers have implemented a way to convey their conception of a given system, different users will interpret it differently.
Their interpretation, being based on individual understanding and knowledge, regulates the users interactions with the system and thereby determines what it really does.
A model of this concept can be seen in \cref{fig:PACT-SystemImage}.

\begin{figure}[H]
	\centering
	\includegraphics[width=0.5\textwidth]{billeder/SystemImage-Benyon.jpg}
	\caption{\textit{The System Image from \citep[p.~31]{Benyon}}}
	\label{fig:PACT-SystemImage}
\end{figure}

The PACT theory presentation and analysis will be written congruently throughout this section.
First the specific theory will be presented followed by the analysis.

\subsection{People}\label{sec:PACT-people}
People differ from each other in many ways.
This element in the PACT analysis is a way to ponder upon these differences as well as a way to categorize the different users of a system.
Three differences usually discussed in this part of the analysis are; \textit{physical, psychological}, and \textit{social}.
In this chapter we will discuss physical and social differences.

Physical differences covers the relevant ways people differ in physical characteristics.
This could be, differences in perception from the five senses; sight, hearing, touch, smell and taste as well, along with age, height or weight.
An example of a physical difference could be color blindness, which affects about $8\%$ of men and $0.5\%$ of women in the world \cite{ColourBlind}.

Social differences is about how people use a system for different reasons and therefore can have various goals.
Additionally, there is also a big difference in people's expertise levels which affects how an interface might be designed.
This is an important consideration since designing a system for a homogeneous group of people is  different from designing one for a heterogeneous group.

In the case described in this report, see \cref{sec:CaseDescription}, there are four main groups of people that need to be considered.
These are the quality manager, secretary, department heads, and blue-collar workers.
This is a heterogeneous group with IT experience ranging from possibly none to what is required at everyday office work.
Should a blue-collar worker have no IT experience it seems reasonable to assume that this blue-collar worker may be cautious of it and the design needs to take this into account.
The group also has different levels of domain expertise ranging from novice to expert which needs to be considered.
The quality manager is a domain expert and therefore has different needs of the system than the blue-collar worker with no expertise.
Furthermore, issues such as colorblindness, other handicaps, as well as bad memory needs to be considered when designing the system and interface.

\subsection{Activities}\label{sec:PACT-actvities}
\todo[inline]{Andreas: Omformuler ``when considering'' PS: bearing in mind}
%Each actor in the system has a different use of it
% Ved ikke om det er mening at den ovenståede sætning skal stå der?

It is important to figure out what activities the system need to support.
First and foremost, the analysis of the activities should focus on the overall \textit{purpose} of the activity.
However each one of the groups described in \cref{sec:PACT-people} has a different goal in using the system.
The secretary,  quality and department managers need to manage, update and access the handbook, whereas the blue-collar workers only need to access and read certain documents.
This will be further elaborated in \cref{sec:Actors}.
The different kinds of purposes an activity can have is explored below.

\textit{Temporal aspects} covers different features in the system as well as a consideration of how the users' interactions with the system are done during a day.
Starting with the considerations of user interactions, one need to consider among other:

\begin{itemize}
	\item How often is the interactive system used?
	\item Is the interactions with the system done continuously or interrupted?
\end{itemize}

For the temporal aspects in a given system the developers need to consider things such as; time pressures, peaks and, and the response time of a given activity.

\textit{Cooperation} is the consideration of whether or not the activity is done in cooperation with others, alone, or a mix hereof, as well as how and when a possible cooperation is needed.
This is an important consideration since a system which is done in cooperation with others needs to have an awareness of all the users, be able to coordinate between them and possibly have a communication system implemented as well.
On the other hand if the activities are done completely alone these features are not necessary.

\textit{Complexity} describes the question of how well-defined the activity is.
Is it a well-defined task which can be done in a simple step-by-step design, or is it a vague task, which would require the users to browse around?

\textit{Safety Critical} has two sides to it.
First is whether or not the activity in itself is ``safety-critical'' where mistakes could be reason for injury or serious accidents.
Second is the consideration of ``what will happen when mistakes and errors are made?'' which is important for any developer to reflect upon.
%Anna: fjerenede sidste del af sætningen (nedenfor) synes ikke den virkede nødvendig
%"and then design for these circumstances."

\textit{The nature of the content} is a more technical reflection of which data requirements are needed for the activity as well as what media it requires.

% Kunne det være en ide at lave en subsubsection ved navnet "Activities of OBHandbook" eller noget da activities del er ret laaaang
%Anna: Hvis vi har tænkt os at gøre det skal det are gøres ved alle dele for at være konsekvente

In the case of this project, see \cref{sec:CaseDescription} in terms of temporal aspects during the activities users may experience interruptions and the system should therefore be simple to use and easy to get back to.

Being easy to use is a feature especially relevant for the blue-collar workers, as they only occasionally need to read documents and therefore are less in contact with the system.
%Anna: har udkommenteret sætningen nedenfor følte den fyldte mere end den breagt noget til bordet
%"This makes it necessary to make the system as easily accessible for them as possible."
The activities would most often be done during business hours, from 8-16 on weekdays, though may be accessed outside of this time frame as well.
\todo[inline]{Rasmus: Skriv om notifikiationer}
% Taniya: kunne måske være en ide at kort nævne administrator at selvom de bliver forstyrret med notifikationer fra systemet så skal admin hurtigt kunne vende tilbage til arbejdsopgaven admin var i gang med.
%Men stadig nemt kunne se/huske/ finde tilbage til de notifikationer, så de hellere ikke bliver glemt.

The cooperation aspect is mostly relevant for the  quality manager and department heads, or in more general terms the administrator and writer roles as they are maintaining the handbook documents.
%Anna: Har udkommenteret sætnignen endenfor da det gav afsnittet et bedre flow
%"Here it is necessary to not only write new versions of a document but also make sure that the newest versions are approved."
Though writing the documents in itself does not require collaboration after a document has been written, it needs to be approved by other users.
This is one of the most important collaboration aspects in the system.
As for the blue-collar worker's activities, there is no cooperation involved as all they need is to read the newest, relevant documents individually.

Regarding the complexity; the activities are quite well-defined overall.
The main activities for the administrative personnel is, that whenever a new document has been written, it needs to be approved and then added to the handbook.
Hereby archiving the the old and now outdated document.
The department manager needs to make sure that the affected blue-collar workers read and understand the new documents.
%Til Astrids beskred, så er det vel som sådan ikke department manageren der holder øje, men kvilitets chefen der kan gå til dem og sige du skal lige tage fat i dine medarbejdere?


In relation to safety critical aspect, there are no physical safety issues to consider in relation to the system, though there are serious consequences to consider if the handbook documents do not live up to regulations.
It is required that the handbook dis up to date and that the elder versions of its documents are stored somewhere.
If this is not upheld the firm could suffer loss of certification, see \cref{sec:standards}, which would result in great loss of revenue.

In terms of the nature of the content, the system should be able to handle Microsoft documents such as Word and Excel as these are the filetypes used to write the documents.
As Ipsen has mentioned, it is also acceptable that the system accepts PDF files instead, as long as she is able to write the document with her preferred text editing software.
%Anna: er det noget vi har kommenteret tidligere at hun ar sagt således?
%Lige nu står det som en selvfølge, er det ikke bedre i stedet for "As Ipsen has mentiond" i stedet for at skrive noget i still med "Although Ipsen has clarified that for it to accept only PDF also is acceptable as long as it is possible to write the documents in one own prefered text editing software"?
Furthermore the system should support large quantities of these files as the archived versions of the documents would be stored indefinitely.

\subsection{Contexts}
Activities always occur in a context which will be explored in this section.
Context can be thought of, as something that surrounds activities as well as a feature which binds them together into a whole.
Usually when considering this point in PACT the three points: \textit{Physical environment, Social context}, and \textit{Organizational context} are at the main focus.
The analysis will not include the social context as this is not considered relevant for the system development.
%Anna: Synes ikke denne tekst stemmer overens med hvad det egentligt er social contest dækker over, på nogen måde
%"This is because the system is developed for a business and work context and does not consider the social interactions of the users for maintaining the handbook documents."

The physical environment might cover everything from weather to geographical placement of where the activity is done.
It is an analysis of the surrounding environment which may have effect on how users perform activities.

There are two main physical contexts to consider.
For the quality manager and the secretary an office context is most likely as they handle administrative work.
For the blue-collar workers the context could differ dramatically as their functions may differ from each other.
The main focus here is that the handbook documents should be easily accessible, no matter in which context the blue-collar workers are located.
Previous to this development project, the most common way that the handbooks were used by employees were in a paper format, which allowed the use of the documents in circumstances not conductive to electronic equipment.
It is expected that this practice will continue after the project.

As for the organizational context there are different factors to consider.
These are Ipsen who acts as a third party quality manager for a firm in relation to maintaining the handbook.
Then there is the in-house secretary who Ipsen works the closest with.
There's the CEO who from time to time reads, writes, and approves the handbook documents.
Lastly there are the blue-collar workers within the firm whose work and daily tasks vary depending on their positions.

It is because of the structure of the organization that there needs to exist different access rights in the system which are administrator, writer, and reader.
The administrator role is mainly for the quality manager who is the main person to manage the handbook documents.
There are actors within the firm, the CEO for example, who occasionally needs to write/update documents in the handbook and need writer access rights.
All of the blue-collar workers need to read and understand the specific documents in the handbook that relates to their work.
These blue-collar workers need reader rights to access the documents.

\subsection{Technologies}
Technologies is the reflection on the medium which interactive system developers work with.
It covers looking into elements such as what medias and technologies are needed to best get input and present the output, and review the communication between the needed devices.
Furthermore, it also examines the content, which concerns the data in the system and its form.
Good content is defined as accurate, up to date, relevant and well presented.

Since the activities are most often done in an office environment, on a standard PC, the input medias are keyboard and mouse.
In addition output is presented through a monitor, or for the blue-collar workers on the factory floor through a printed version of the handbook, or parts of it.
The content displayed to the user, on the monitor, is presented through a web browser.
In \cref{sec:conditions} it was mentioned that the majority of the users have limited IT-experience.
It is therefore important that the content displayed is simple enough to allow the users to easily navigate and use the system.

Furthermore, the user-to-user communication is usually done through telephone or emails, while it between devices is done over a network, such as the internet.
Lastly, the system-to-user communication is done through notifications.
These will be sent every time a document, relevant to the user, has been updated and the user therefore has to read it, or when the user has been assigned to approve a new version.


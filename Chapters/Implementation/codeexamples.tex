\section{Code examples}

In this section examples of the code will be presented.
As written in \cref{sec:systemdesign} the system is based on the ASP.NET Core MVC architecture including EF Core.
How the system utilizes the MVC architecture and the EF Core framework will be shown in the following code examples.

Firstly the MVC implementation is explored with the Document class.
How the Document object is being treated throughout the system will be presented throught the model, view, and controller components.
Furthermore it will be shown how the document data will be stored in the database with the repository pattern and EF Core.

\subsection{Model}

\begin{lstlisting}
[Key]
public int ID { get; set; } //for the database

public virtual Chapter Chapter { get; set; } //first part of doocument ID

public int ChapterNumber
{
	get
	{
		return Chapter != null ? Chapter.Number : -1;
	}
}

[Range(1, Int32.MaxValue)]
[Display(Name = "Section")]
public int SectionNumber { get; set; } //second part of Document ID

[Required]
public string Title { get; set; } // TODO: Unique title of document

public virtual ICollection<Version> Versions { get; set; } //List of all past versions of the document (the "archive")
public virtual IEnumerable<DocumentDepartment> DocumentDepartments { get; set; }

public bool Archived { get; set; }

public Document()
{
	Archived = false;
}

\end{lstlisting}

\begin{lstlisting}
public void AddVersion(Version version)
{
	version.ValidFromDate = DateTime.Now;
	version.Approved = true;
	Versions.Add(version);
	int nVersions = Versions.Count;
	var oldVersion = Versions.ElementAtOrDefault(nVersions - 1);

	if (nVersions >= 2 && oldVersion != null)
	{
		oldVersion.ValidUntilDate = DateTime.Now;
	}
	foreach (DocumentDepartment dd in DocumentDepartments)
	{
		dd.Department.Notify("The document " + Title + " has been updated", "http://localhost:5000/document/" + Chapter.Number.ToString() + "." + SectionNumber.ToString());
	}
}
\end{lstlisting}

\subsection{Controller}

\begin{lstlisting}
[HttpGet("")]
[HttpGet("document/")]
public async Task<IActionResult> Index()
{
	var user = await this._userManager.GetUserAsync(HttpContext.User);
	var documentIndex = new ViewModels.DocumentIndex();
	foreach (var ud in user.UserDepartments)
	{
		foreach (var dd in ud.Department.DocumentDepartments)
		{
			if (!dd.Document.Archived) documentIndex.AssignedDocuments.Add(dd.Document);
		}
	}
	List<Document> documentList = await this._documentRepository.ListNonArchivedAsync();
	foreach (var d in documentIndex.AssignedDocuments)
	{
		documentList.Remove(d);
	}

	foreach (var d in documentList)
	{
		if (!documentIndex.UnassignedDocuments.ContainsKey(d.Chapter))
		documentIndex.UnassignedDocuments[d.Chapter] = new List<Document>();
		documentIndex.UnassignedDocuments[d.Chapter].Add(d);
	}

	@ViewData["Title"] = "Table of Contents";

	return View(documentIndex);
}




\end{lstlisting}

\begin{lstlisting}
[HttpPost("document/add/")]
public async Task<IActionResult> Add(Document document,
	int chapterNumber,
	IFormFile versionFile,
	IFormFile workingFile,
	string requireApprovalCheck,
	string approvers)
{
	bool requireApproval = (requireApprovalCheck == "on");
	@ViewData["Title"] = "New document";
	@ViewBag.Chapters = await _chapterRepository.GetChapters();

	if (ModelState.IsValid)
	{
		document = this._documentRepository.ToProxy(document);
		var chapter = await this._chapterRepository.GetChapter(chapterNumber);

		if (chapter == null)
		{
			ModelState.AddModelError(string.Empty, "Problem happend getting the chapter.");

			return View(document);
		}

		document.Chapter = chapter;

		try
		{
			document = await this._documentRepository.AddAsync(document);
		}
		catch (DocumentAddException e)
		{
			ModelState.AddModelError(string.Empty, e.ModelError);
			return View(document);
		}
		if (versionFile != null)
		{
			OBHandbooks.Models.Version version = new OBHandbooks.Models.Version();
			version.SetDocument(document);
			version = await this._versionRepository.AddAsync(version);

			version.VersionFile = new HandbookFile(versionFile);
			if (workingFile != null) version.WorkingFile = new HandbookFile(workingFile);

			if (requireApproval)
			{
				HandbookApproval approval = await CreateApproval(
				approvers,
				await this._userManager.GetUserAsync(HttpContext.User));
				approval.DocumentVersion = version;
				version.DateSubmittedToApproval = DateTime.Now;
				document.Versions.Add(version);
				approval.CheckForApproval();
				await this._approvalRepository.UpdateAsync(approval);
			}
			else
			{
				document.AddVersion(version);
			}
			await this._documentRepository.UpdateAsync(document);
			await this._versionRepository.UpdateAsync(version);
		}
		return RedirectToAction(nameof(Index));

	}

	return View(document);
}

\end{lstlisting}

\subsection{View}

\subsection{Repositories}

\subsection{Entity Framework Core}
Entity Framework Core is used to communicate with the database, which on it's own is not object-oriented.
For more information on EF Core see \cref{sec:efcore}.
\subsubsection{Postgresql}
Postgresql was chosen as the database for the project.
The decision was purely based on prior experience with setting up the database.
Postgresql is often better suited for more advanced applications, [CITATION NEEDED] but as this application mostly used CRUD operations with few to no complicated queries, this was hardly relevant.
\section{Repository Pattern}
In this project the repository pattern was used. It abstracts communication with the database, allowing for easily replacing it if necessary, and mocking it during unit tests.

The reposito




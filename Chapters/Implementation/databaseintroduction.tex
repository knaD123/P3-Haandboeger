This chapter is based upon information from \url{https://www.oracle.com/database/what-is-database.html}.

A database is a combination of a \textit{Database Management System} (DBMS), a collection of structured data and the associated computer system(s).
A DBMS is a piece of software that serves as an interface between the database and the users or the computer system(s) that allows them to perform \textit{create, read, update and delete} (CRUD) operations on the database as well as manage how the data is organized. It also allow them to do some more administrative operations e.g. performance monitoring, and backup and recovery.

Some of the most popular DBMSs today are: 
\begin{itemize}
        \item MySQL
        \item MSSQL (Microsoft SQL Server)
        \item PostgreSQL
        \item Oracle Database.
\end{itemize}

The most common databases today structure the data in one or more tables where each table consists of one or more rows and columns. Structuring the database like this makes it easy to perform CRUD operations on the database using a \textit{structured query language} (SQL). SQL is a programming language that is used by most \textit{relational databases} to perform CRUD operations.

% Henrik: skal nok lige tjekkes / omformuleres
One might be tempted to ask \textit{"but why not just use a spreadsheet?"} (e.g. Microsofts Excel). Spreadsheets are also a great way of storing data but spreadsheets were originally designed for a single user that did not have to do a lot of complicated data manipulation.

Not all databases are relational databases, which became popular in the 1980's, but there also exists e.g. \textit{object-oriented databases} and \textit{NoSQL databases}. \todo{Skriv noget smart om OO-databases og NoSQL databases}

\section{Web Frontends}
The languages of UI on the web are HTML, CSS, and Javascript, and each of these three languages have their own respective roles.\cite{nixonweb}
\subsection{HTML}
HTML is the most important of the three.
It's a markup language that is used to describe the structure and much of the contents of a webpage.
\begin{lstlisting}[language=HTML]
<!DOCTYPE html>
<html>
<head>
<title>HTML Example</title>
</head>
<body>
<h1>HTML</h1>
<p>This is a <b>webpage<b></p>
</body>
</html>
\end{lstlisting}
The different tags can contain child tags, which in turn can have their own tags.
This gives the webpage a tree structure.

Two important types of tags are the \texttt{<script>} and \texttt{<style>} tags.
These allow for embedding CSS and Javascript into the page.
The CSS allows for different kinds of styling, such as colouring and positioning of tags.
\subsection{CSS}
CSS is short for Cascading Style Sheets. It allows for defining the style of the document.
\begin{lstlisting}
body {
	background-color: green;
}

p {
	text-color: blue;
}
\end{lstlisting}
This stylesheet, for example, colours the background green, and the text in paragraph (\texttt{<p>}) tags blue. The possibilities with CSS are of course far greater than just colours, as the language also allows for positioning elements, changing their visibility, and far more features than can be covered in this project.\cite{nixonweb}
In working with CSS, we will primarily be working from an existing CSS Framework called Bootstrap. Having Bootstrap, which predefines a lot of web components, greatly eases the development of web applications. While this naturally comes with a tradeoff for customizability, it was concluded that this project did not need any cutting edge UI to achieve its goals.
\subsection{Javascript}
Javascript is a scripting language built for the web.
Javascript allows for running code on the page, which can interact with the contents of the page, play sounds, send requests in the background, and much more.\cite{nixonweb}

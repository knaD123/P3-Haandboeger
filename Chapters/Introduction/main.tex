\chapter{Introduction}
\section{Initial problem statement}
\section{Work method}
The strategy is selected according to the principles presented in the OOA\&D method. This method puts great emphasis post-poning non-critical decisions so as to maintain the freedom to make changes once more information is available and on working on the areas that pose the greatest challenge first. %Anja: Præsenter OOA&D metoden før du nævner den.

% I don't understand the next sentence here
% Taniya: does the set of questions have a name?
A set of questions supposed to serve as a starting point for determining which area that is, is provided in the book [howdidwedothisagain]. The questions are answered so that a yes equals to zero points, a maybe to one point and a no to two points. If the answer is unknown it is counted as a no. The more points an area has the more critical the corresponding area is. The list of these questions and answers can be found in appendix XX.
% Henrik: Synes ikke der burde var BLANK linie her, da de to afsnit hænger sammen
Based on this activity it was concluded that AAA is the most difficult area, with BBB as number two and CCC last.
% Maybe a sort of chapter division here?
% Anja: Generelt skal den ovenstående tekst stå et andet sted. Informationen virker lidt malplaceret.
Selection of a general work structure depends on the degrees of structure and complexity in the project. The waterfall model, where one part of the project is completely finished before the next one is begun is great for projects with low uncertainty and high complexity. However an iterative approach where every part of the project is revisited  multiple times is better for projects with high uncertainty and low complexity. [how does one refer to things the teacher has said but which are not in his own damned book?] % I think you have to find a book that says it

The user is someone who works with handbooks in a professional setting and therefore either has or quickly gains a lot of experience. The task is very structured and one of the main purposes of the system is to make sure that the books remain structured in the face of human errors. The developers on the project however are students with limited experience, which is what provides the most uncertainty to this project. %Anja: Jeg synes ikke, at vi skal skrive, at grunden til at, der er "uncertainty" i projektet pga., at vi er studerende. Det er mere fordi at selve designprocessen i sig selv er usikker/omskiftelig. Hele grundlaget for at have en 'design'-proces til at starte med er, at vi ikke ved hvordan slutproduktet skal se ud på forhånd; derfor er projektet "uncertain".
% Henrik: Hvad er det "the developers" har limited experience med?

The size of the project can be adjusted until it fits the length of the project period and the amount of users is relatively low: There is only one main user and a few who are not as involved. The amount of the new information is very high at the beginning of the project but once understood the complexity is insignificant.

As both the uncertainty and complexity of the project are relatively low, both methods can be used. As there is a high complexity in the very beginning of the project the waterfall method will be used here with an iterative process in the remaining parts of the project to make sure the terms were not misunderstood and to deliver the best possible system to the users. %Nævn gerne noget med designprocessen her.
%Anja: Ellers er teksten fin (y)
% Henrik: Måske flyt dette afsnit op sammen med det tidligere afsnit, hvor der snakkes om "unvertainty and complexity"?

\todo[inline]{Vi skal måske have snakket om placeringen af denne tekst igen. Jeg synes, at den kommer lidt malplaceret ift. indledning. Jeg synes, at det giver mere mening efter problemanalysen.}

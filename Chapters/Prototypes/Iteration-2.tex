\section{Second iteration}



\subsubsection*{FACTOR}
There are two big differences between the final FACTOR analysis and the preliminary one.
First is that during the first iteration the role hierarchy was turned from three levels into four. 
In the original FACTOR it was written as follows
\footnote{Note: The writing style was more detailed in the first FACTOR to make sure the requirements was understood correctly, when the second interview came around.}:
\newline
%Anna: HVordan tydeliggøre mman at vi citere noget tidligere skrevet (skrives det i italics eller skal placeringen være anderledes eller er der andre trics?)
''Different levels of permissions/access rights to the documents within the handbook
\begin{itemize}
	\item 
	0 level:
	Reading the handbook except secured documents
	\item 
	1st level:
	Reading the whole handbook
	\item 
	2nd level:
	Editing selected documents
	\item 
	3rd level:
	Total access to editing documents and access to archive''
\end{itemize}

The second difference is how most of the functionality and some of the conditions from the FACTOR analysis has turned into requirements in the MoSCoW presented in \cref{sec:requirementsdefinition}. These are elements such as,
\begin{itemize}
	\item From conditions
	\begin{itemize}
		\item 
		''The system needs to handle several different file types \ldots''
		\item 
		''The handbook should be printable''
	\end{itemize}
	\item From functionality
	\begin{itemize}
		\item 
		''Title and chapter number are linked. If a title is removes the chapter number cannot be used for anything else.''
		\item 
		''There needs to be a Table Of Contents with title, chapter number and date/version number''
	\end{itemize}
\end{itemize}

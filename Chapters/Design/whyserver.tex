\subsection{Web Applications}
The requirements in the system definition (see \cref{sec:systemdefinition}) state that users should be able to work together through the system, independent of geography.
The OOA\&D method designates that in these scenarios, the client-server model should be considered. \citep[p.~202]{Rod-Aalborg}

Communication between clients is required for features such as:
\begin{itemize}
\item Having multiple users being able to access the same handbook
\item Registering whether the document has been read
\item Being certain you have the newest version
\end{itemize}

The possible forms of communication are, in broad terms, a decentralized approach or a centralized approach.
It is, however, pretty clear that a centralized approach is ideal.
A decentralized system is more complex to develop and manage, and this project, presents tangible benefit.

What is left to decide is what sort of technology is used to implement the client-server architecture.
The two possibilities are roughly; a web application, or a desktop application that speaks to a server.
While a desktop application is often faster, and has bigger opportunities for interacting with the hardware, it is also harder to manage on the users computer and keep the application up to date on each machine.
Whereas, the application logic for a web-app will be constantly updated, as the part of the application that is currently used is downloaded to the users computer as a part of each request.
The web-app can be thought of as an additional layer of abstraction that does away with platform specific issues, such as adapting to different operating systems.
A disadvantage of web-apps is the requirement of there being a server, and the possible loss of the ability to work if it is down.
However, since the application already requires a server, this is a smaller problem.

The largest disadvantage, and quite possibly one that has not been considered enough in this development project, is the learning curve associated with web applications.
The web builds on a lot of technologies, CSS, JavaScript and HTML, which are absolutely necessary, but can be difficult to learn all at once, which will increase development time.

The model decided upon here is that of local presentation \citep[p.~202]{Rod-Aalborg}.
As the integrity of the data is incredibly important, all business logic should be centrally administrated, and that way be resilient to user error.
Choosing a local presentation model allows for creating a web application.
A web application has many advantages, including but not limited to, mature user interface (UI), easy cross platform support, and avoiding having to install anything.

A web application is any form of application where the client is a web browser.

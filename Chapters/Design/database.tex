\documentclass[../../master.tex]{subfiles}
\begin{document}
\subsection{Database}
Jan Paredaens et al. \cite{RelationalDatabaseModel} defines a database system as a collection of programs that run on a computer and that help the user to get information, to update information, to protect information, in general to manage information.

Overall, there exist three different types of database \cite{RelationalDatabaseModel}:

\begin{itemize}
    \item
    \textit{Network model} where the structure of information is represented by a directed graph.
    \item
    \textit{Hierarchical model} where the information is represented as a set of trees.
    \item
    \textit{Relational model} where the information is represented in tables.
\end{itemize}

Relational databases are one of the most popular where, as mentioned above, the data is organized in tables with rows and columns.\cite{OracleWhatIsDatabase}
The application discussed in this report is using such a type of database due to it being the best supported in ASP.NET Core.

To get a better understanding of how the data is structured in a relational database imagine a system with a lot of users.
This system might be interested in keeping track of the users \textit{username, password, firstname, lastname} and \textit{email}.
In a relational database this information can be represented as follows:

\begin{table}[H]
    \centering
    \begin{tabular}{lllll}
        USERNAME & PASSWORD & FIRSTNAME & LASTNAME & EMAIL \\
        \hline
        username1 & pw123 & Billy & Johnson & bj@mail.com \\
        username2 & pw456 & Lucy & Johnson & lj@mail.com \\
    \end{tabular}
    \caption{Table of users}
\end{table}

Structuring the users like this makes it easy to add new users to the table since the number of columns does not change, nor their names.

The most common way to interact with a database is to use SQL.
SQL is a database sublanguage that differs from other computer languages because it describes what the computer \textit{should do} rather than \textit{how it should do it}.\cite{SQLIntroduction}

SQL makes it easy to perform Create, Read, Update, Delete (CRUD) operations on the database.\cite{OracleWhatIsDatabase}
An example of how to use SQL could be someone who wants to retrieve complete infromation about all the users from the user table

\begin{lstlisting}[caption={An SQL query that fetches complete information for every user}, label=lstSQL-user1]
SELECT * FROM USER;
\end{lstlisting}

Or maybe this person is just interested in retrieving a specific user

\begin{lstlisting}[caption={An SQL query that only fetches complete information for username1}, label=lstSQL-user2]
SELECT * FROM USERS WHERE username = "username1";
\end{lstlisting}

It has now been explained what a database is, how it works and how to interact with the database.
The next section will move from backend technologies, to explaining frontend technologies.

\end{document}

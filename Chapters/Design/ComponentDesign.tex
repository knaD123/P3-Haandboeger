\section{Component Design} \label{sec:componentdesign}

In this section the component design theory will be introduced and elaborated.
This theory will be utilized in the follwing section where the component design for the system will be presented.

\subsection{Component Design Theory} \label{sec:archicomponents}
The component design \cite{Rod-Aalborg} is the structural view of the system concerns, where the each of the components are being viewed in relation to their specifications and responsibilities within the system.
A component is defined, \citep[p.~192]{Rod-Aalborg}, as.
\begin{defn}\label{defn:component}
	A collection of program parts that constitutes a whole and has well-defined responsibilities.
\end{defn}

These components when combined will make up the component architecture which defines the system structure, \citep[p.~192]{Rod-Aalborg}, see the following definition.
\begin{defn}\label{defn:Structure}
	A system structure is composed of interconnected components.
\end{defn}

\subsubsection*{The Layered Architecture Pattern}

The layered architecture design means that the components are designed in layers such that the components are above or below each other.
Here each layer or component describes its responsibilities and which other components it can access.
This also means that when one component is being changed, then the other components down the layers will likewise change.
The component layers does not exclusively have to be vertical as they can also be designed with horizontal decomposition which denotes sub components.

The layered architecture can be distinguished between open and closed, strict and relaxed architecture which provides four combinations:

\begin{center}
	\begin{tabular}{| c | c |}
		\hline
		Closed-Strict & Open-Strict \\
		\hline
		Closed-relaxed & Open-Relaxed \\
		\hline
	\end{tabular}
\end{center}

A \textit{closed} architecture means that the layer can only access components which are immediately adjacent to it thus it can only access those that are directly above or below it.
An \textit{open} architecture means that the component can access any other component no matter how far above or below it is.
A \textit{strict} archicture is only able to access components that are either above or below, but not both.
Whereas, the \textit{relaxed} architecture is able to access components from both directions.

\subsubsection*{Interface, Function, And Model Components}

When designing the component architecture there can usually be defined \textit{interface, function}, and \textit{model} components.
Each of these have their own role and responsibilty within the system.

''A model component's main responsibility is to hold the objects that represent the problem domain. '' \citep[p.~203]{Rod-Aalborg}.
In other words the model component should seek to solve the problems of the problem domain.
Whenever something changes in the problem domain, the model component should be changed accordingly to solve the new problem that has arisen.

''The main responsibilty of a function component is to provide the model's functionality'' \citep[p.~205]{Rod-Aalborg}.
This component ensures that the solutions and algorithms from the model component is accessible for the user interacting with the system.
''The main responsibilty of an interface component is to handle the interaction between the actors and the functionality'' \citep[p.~207]{Rod-Aalborg}.
This component is what the users directly interacts with to access the underlying function and model.


\subsection{System design} \label{sec:systemdesign}

In relation to the ASP.NET MVC pattern there are immediate similarities to the component design theory in \cref{archicomponents}.
There can be drawn parallels between the \textit{model} component and the \textit{model} classes in the MVC pattern.
There can likewise be drawn parallels between the \textit{functionality} component and \textit{controller} classes, and \textit{interface} component and the \textit{view} in the MVC pattern.
The parallels are listed as seen in the table below:

\begin{center}
	\begin{tabular}{| c | c | }
	\hline
	\textbf{Components} & \textbf{MVC Pattern} \\
	\hline
	Model & Model \& EF Core \\
	\hline
	Functionality & Controller \\
	\hline
	Interface & View \\
	\hline
\end{tabular}
\end{center}

As written in the component design theory the model components should seek to solve the problem of the problem domain.
The model component in the MVC pattern is likewise going to contain the classes and objects that seeks to solve the problem domain through meeting the requirements written in section xx.
The model component in the implementation is among other classes going to contain classes for \textit{documents, users, departments, versions, approvals,} and \textit{read statuses}.
The arguments for the inclusion for these classes can be seen in \cref{sec:classdiagram}.
The database in form of EF Core is also included in the model component as this in part also seeks to solve the problem domain.

As written in the component design theory the responsibilty of the functionality is to provide the actor a means to access the model component.
The controller component in the MVC pattern does this by communicating between the model component and the view component.
The controllers retrieves data from the database and methods/objects from the models and links these to the view.

The responsibility of the interface component is the handle the interaction between the users and the functionality.
This is done through the view model in the MVC pattern as the interface is handled in big part due to html, css and javascript.
These are what determine what the actors see and what they can interact with.

\subsection{Component layers}
In relation to the theory written in \cref{archicomponents} the components can be designed in layers to describe their responsibilities and relation to each other.
The components in the MVC pattern with EF Core included can also be designed with the layered architecture in mind.
To give an overview and understanding of the architecture and design a simple component layer design can be seen in the figure below:

\begin{figure}[H]
	\centering
	\includegraphics[width=0.7\textwidth]{billeder/simplecomponents.jpeg}
	\caption{Simple component layer design}
\end{figure}


View, controller, model, and EF Core are thus the main components layers that can be found in the design, each with a well-defined responsibility.
Ideally the architecture is designed so that the view and model components do not have to interact with each other.
It is the intention that the controller is the link between these components, which is why it is placed in the middle of the layer components.
The main function of the controller is to link objects from the model component and data from the EF Core database and make these accessible for the view component.

This will not always be the case as with the ASP.NET Core MVC model is designed so that a view has a corresponding controller.
For example the document view will have a corresponding document controller.
The document view will eventually have to borrow objects and data from other models in the system, which means that the view will at times have to bypass the controller component to communicate with the model component.

Subsystems can be explored from the main components, which for example are the interface system and its underlyding technologies that it consists of.
These are html, css, javascript, and razor.
The model component has two main parts which are the model classes from the MVC pattern and the EF Core database system.
These two are defined as seperate components which share similary responsibilities in the model component.

These components their relations with each other, and their underlying subcomponents can be seen in the complex version of the layer design below:
\todo[inline]{Fix tabellen efter Lu's kommentarer}

\begin{figure}[H]
	\centering
	\includegraphics[width=1\textwidth]{billeder/complexcomponents.jpeg}
	\caption{Complex component layer design}
\end{figure}

Ideally the design would have \textit{closed-relaxed} architecture as the components would only be able to access layers adjacent to them.
In this design the design would be \textit{relaxed} as the controller would have access to both the view and the model components.
As it is though the design has an \textit{open-relaxed} architecture as the view component occasionally has access to the model component.


In the final design there will be several classes included within each of the components.
Each of the main classes in the model component will have a corresponding controller and view in relation to it.
For example the document class in the model component will have a corresponding database in EF Core and corresponding controller and view.
Here the documents object is within the model component and the documents data will be stored in the database.
A corresponding controller, which mainly handles the document class, ensures that the class and database are accessible for the actor.
A corresponding view ensures that the actor is able to see and interact with the document classes and database.

\subsection{MoSCoW rules}\label{sec:requirements}
It is important to prioritize a system's requirements after the first draft of a system has been designed.
This is because there are no guarantees that all of the requirements and design ideas are doable within the time frame of a given project.
The information in this chapter is based upon Benyon \cite{Benyon}.

These priorities are set to determine which requirements are absolutely necessary for the system to function, the \textit{minimal viable product} (MVP),
and which requirements are simply nice to include.
The \textit{MoSCoW rules}
have been chosen to help prioritize the requirements for this project.
The MoSCoW rules classifies these priorities into:

\begin{itemize}
    \item \textit{Must have}.
    \item \textit{Should have}.
    \item \textit{Could have}.
    \item \textit{Want to have but Won’t have this time around}.
\end{itemize}

Must haves are the fundamental requirements to include where system would not function without.
Must haves also determines the MVP.
Should haves are essential requirements to include if the time frame allowed, but the system could still function without.
Could haves are less important requirements than must haves and should haves and they would easily be excluded from the system.
Want to have but Won't haves are requirements that can wait until a later point of the system development.

For the sake of preventing misunderstandings the definition for requirement used in this report \citep[p.~147]{Benyon} can be found in \cref{defn:Req} below.

\begin{defn}\label{defn:Req}
    A requirement is something the product must do or a quality that the product must have
\end{defn}

\subsubsection{Defining requirements} \label{sec:requirementsdefinition}
It was in \cref{sec:PACT} mentioned how it is important, when designing an interactive system, to do it with the users in mind.
Therefore, before defining a project's requirements it is essential to figure out the client's wants and needs.
A lot of different techniques can be used when defining a project's requirements, e.g.:

\begin{itemize}
    \item Interviews
    \item Observing people and record their activities on video
    \item Organizing focus groups
\end{itemize}

These techniques can help designers to better understand the problems the client has with their current system, and at the same time, give a better understanding of the requirements for the new system.
Using these techniques where the designer interacts with other people also provides them with a lot of stories that can form the basis for the analysis work.
Even though a designer have all these techniques at their disposal, the interview technique
is one of the most effective techniques to find out what the client wants and needs.

The process of defining a project's requirements is an iterative process, see \cref{sec:iterativModel}.
This is because new requirements will pop-up throughout the design process.
This may happen when the user interacts with the system during tests and it becomes clear misunderstandings of the basic concept has happend.
Furthermore, it may also happen when the user realises other needs than what was thought necessary of the system to begin with.

For this project the interview technique was used to define this project's requirements, see \cref{sec:firstinterview}. This helped defining the prioritization of the following requirements:

\begin{itemize}
    \item
    Must have:
        \begin{itemize}
            \item
            The system must manage versions of handbook documents
            \item
            A \textit{Table of Contents} (TOC) with title, ID number, date and version number
            \item
            The system must be able to handle PDF files
            \item
            Title and ID number are linked
            \item
            Once a version has been added to the handbook, it cannot be changed
        \end{itemize}
    \item
    Should have:
        \begin{itemize}
			\item
			Automatic updates of TOC
            \item
            A human-written changelog may be depending on the company settings
            \item
            Registration of who among specific employees have read a version
            \item
            The handbook should be printable.
            \item
            Different levels of access rights to the documents within the handbook
            \item
            Readers and writers only have access to the newest version of a document
            \item
            Administrators have acces to all features supported by the system, except marking versions as read
            \item
            It should be possible to group users into departments, and associate them to documents.
            \item
            It should be simple to switch from an exsisting system, and back again
                \todo[inline]{RASMUS: Tjek om den her formulering passer med FACTOR, omformuler til kvantitativt krav}
        \end{itemize}
    \item
    Could have:
        \begin{itemize}
            \item
            When a document is updated, a notification could be sent out to the associated departments.
            \item
            The system could be able to handle documents of different file-types
            \item
            Option to sort documents according to different attributes
            \item
            System for approval of new versions of documents
			\item
			Be able to fill a PDF header with information such as approvers and valid date
        \end{itemize}
    \item
    Want to have but Won't have this time around:
        \begin{itemize}
            \item
            Highlight differences between current and previous versions.
            \item
            Approval of suppliers and their documents.
        \end{itemize}
\end{itemize}

The requirements belonging to this category are necessary since the system would not be functional without them.
The first item under \textit{must have} which is ''The system should manage versions of handbook documents'' is intentionally broad as it is expanded upon in the following four items.

The following prioritization \textit{Should have} contains very important requirements that the system must include if Ipsen were to use the solution.
The reason that these requirements are placed here is because the system is technically able to function as a management software without these specific requirements.

The last priorizations \textit{Could have} and \textit{Want to have but Won't have this time around} are the least important requirements in relation to the system.
These items are placed under these priorities as they would be nice to include in the system, but Ipsen does not consider them vital to the solution or her problem.

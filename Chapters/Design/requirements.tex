\section{MoSCoW rules}
The information in this chapter is based upon information from Designing User Experience by David Benyon \cite{Benyon}.
%astrid: jeg har smidt titlen ind fordi jeg tænker at enten introducerer man bogen eksplicit titel og det hele eller også introducerer man den slet ikke.

It is important to prioritize a system's requirements after the first draft of a system has been designed.
This is because there are no guarantees that all of the requirements and design ideas are doable within the time frame of the project.
There is in this project set a hard time limit by Aalborg Univerity.

These priorities are set to determine which requirements are absolutely necessary for the system to function, the \textit{minimal viable product} (MVP), and which requirements are simply nice to include.
The \textit{MoSCoW rules} have been chosen to help prioritize the requirements for this project.

The MoSCoW rules classifies these priorities into:

\begin{itemize}
    \item \textit{Must have}.
    \item \textit{Should have}.
    \item \textit{Could have}.
    \item \textit{Want to have but won’t have this time around}.
\end{itemize}

Must have are the fundamental requirements to include of which the system would not function without.
Must have also determines the MVP.
Should have are essential requirements to include if the time frame allowed, but the system could still function without.
Could have are less important requirements than must have and should have and they would easily be excluded from the system.
Want to have but Won't have are requirements that can wait until a later point of the development of the system.

% Henrik: Er dette en okay måde at lave denne definition? DEB-bogen s. 147 øverst
A definition of a requirement is provided to prevent any misunderstandings throughout this chapter.
Requirements can be divided into two types, \textit{functional} and \textit{non-functional}, and a definition for those types will be provided aswell.

\begin{defn}
    A requirement is something the product must do or a quality that the product must have
\end{defn}

\begin{defn}
    A functional requirement is a requirement that the system must do
\end{defn}

\begin{defn}
    A non-functional requirement is a quality that the system must have
\end{defn}

% Henrik: Noget om hvordan man finder frem til disse "requirements"
\subsection{Defining requirements} \label{sec:requirementsdefinition}
% vigtigt at finde ud af hvad "the client(s)" ønske, før man sætter "requirements"
It was in \cref{sec:PACT} mentioned that it is important, when designing an interactive system, to do it with the human in mind.
Therefore, before defining a project's requirements it is a good idea to figure out what the client wants and needs because it is the client, and not the designer, that is going to use the final system.
A lot of different techniques can be used when trying to define a project's requirements, e.g.:

\begin{itemize}
    \item Interviews
    \item Observing people and record their activities on video
    \item Organizing focus groups
\end{itemize}

These techniques can help the designer to better understand the problems that the client has with the current system, and at the same time, give a better understanding of the requirements for the new system.
Using these techniques where there designer interacts with other people also provides the designer with a lot of stories that can form the basis for the analysis work. Even though a designer have all these techniques at their disposal, the interview technique, see \cref{sec:interview}, is one of the most effective techniques to find out what the client wants and needs.

% det kan være en iterativ process, og det sker ofte at der kommer nye "requirements" fra tid til anden
The process of defining a project's requirements is an iterative process, see \cref{sec:iterativ}. This is because new requirements will pop-up throughout the design process.

% Henrik: Noget om hvordan vi er kommet frem til vores "requirements"
For this project the interview technique was used to define this project's requirements, see \cref{sec:firstinterview}. This helped defining the following requirements:

\begin{itemize}
    \item Must have:
        \begin{itemize}
            \item The system should manage versions of handbook documents
            \item A \textit{Table of Contents} (TOC) with title, ID number, date and version number
            \item The system should be able to handle PDF files
            \item Title and ID number are linked
            \item Once a version has been added to the handbook, it cannot be changed
        \end{itemize}
    \item Should have:
        \begin{itemize}
                        \item Automatic updates of TOC
            \item A human-written changelog can be included with each version
            \item Registration when a version has been read by an employee
            \item The handbook should be printable
            \item Different levels of permissions/access rights to the documents within the handbook
            \item Readers and writers only have access to the newest version of a document
            \item Administrators have access to everything
            \item It should be possible to group users into departments, and associate them to documents
            \item It should be easy to switch from an existing system, and back to the existing system
                \todo[inline]{RASMUS: Tjek om den her formulering passer med FACTOR, omformuler til kvantitativt krav}

        \end{itemize}
    \item Could have:
        \begin{itemize}
            \item When a document is updated, a notification should be sent out to the associated departments
            \item The system should be able to handle documents of different filetypes
            \item Option to sort documents according to different attributes
            \item System for approval of new versions of documents
        \end{itemize}
    \item Want to have but Won't have this time around:
        \begin{itemize}
            \item Highlight differences between current and previous versions
            \item Approval of suppliers and their documents
        \end{itemize}
\end{itemize}

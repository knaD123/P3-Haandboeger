%Anna: Burde denne section ikke være en under section til web-applications?
\subsection{ASP.NET Core} \label{sec:aspnetcore}

ASP.NET core is a web framework developed by Microsoft for developing web applications in the .NET Core Framework \cite{aspnetcore2}.
ASP stands for \textit{Active Server Pages} and offers a variety of application models, one of which being the MVC pattern.
The framework has built-in support for many common features, such as database integration, identity and authorization with multi-factor, external authentication and CSS frameworks.

The reason for this web API to be used is because of the semester's requirement to develop an application using the programming language C\#.
%Anna: Er det en lovlig begrundelse i projektet, (Plejer alle andre steder fra jeg ellers har hørt (også sidste semester) at få at vide det er en undskyldning som ikke må bruges men skal forklares med andre argumenter)
ASP.NET Core supports, among others, C\# and is able to integrate the language with a web application.
%Anna: er asp.net ikke et specifikt framework indenfor c#.
%Anja: Den supporter ogsaa et par andre sprog.
It uses a Model-View-Controller design pattern as a basis for the architecture of the applications developed in it.
%Anna: evt henvise til hvor det design pattern diskuteres eller nævnes?
%Anja: Er det virkelig nødvendigt, naar MVC bliver forklaret lige bagefter?

%ASP.NET Core is a web framework developed by Microsoft, for developing web applications in the .NET Core framework.
%The framework is quite widely used, which means that there are plenty of modules available for doing many jobs. Besides that, the framework has built-in support for many common features, such as database integration, identity and authorization with multi-factor and external auth, CSS frameworks, and much more.
%It uses the quite popular Model-View-Controller design pattern as a basis for the architecture of the applications developed in it.

\subsubsection{Model-View-Controller design pattern}
The Model-View-Controller pattern is an architectural pattern that splits an application into three basic components; the model, the view, and the controller.
Originally meant for GUI (Graphical User Interface) development, the pattern has been adapted into web development, and is used by many of the frameworks in the space. \cite{gangoffour}
%Anna: Er forkortelsen for GUI tidligere blevet nævnt ellers skal vi måske overveje at skrive det ud og så give forkotelsen?

The advantages of using the MVC architecture are that it helps with the complexity of the application by dividing its components into three individual components each with its own function and responsibilites.
%Anna: ser gerne vi skriver "helps simplifiying the complexity" fremfor "helps with the complexity"
These components are \textit{model, view}, and \textit{controller} which will be elaborated below.
%Anna: evt skrive "These components are as mentioned before \textit{model, view}, and \textit{controller}, see \cref{fig:MVC-components}, which will be elaborated below."
Furthermore the MVC archictecture supports test driven development and also does not use server-based forms which gives the developers full control over the application. \cite{mvcarticle}
%Anna: evt værd at påpege vi ikke har lavet test driven development da det kræver vi designede testene før vi skrev de enkelte program dele

\begin{figure}[H]
\centering
	\begin{tikzpicture}[node distance=2cm]
	\node[process] (model) {Model};
	\node[process] (view) [below of=model, left of=model] {View};
	\node[process] (controller) [below of=model, right of=model] {Controller};
	\draw[arrow] (model.south west) -- node[left] {Updates} (view.north);
	\draw[arrow] (controller.north) -- node[right] {Updates} (model.south east);
	\node[process] (user) [below of=model, node distance=3.5cm] {User};
	\draw[arrow] (view.south) -- node[left] {Perceives} (user.north west);
	\draw[arrow] (user.north east) -- node[right=0.2cm] {Interacts with} (controller.south);
	\end{tikzpicture}
	\caption{The Model-View-Controller design pattern.}\label{fig:MVC-components}
\end{figure}

As mentioned the three main components in the architecture are model, view, and controller as shown in the figure above.
%Anna: nu er det tredje eller jerde gang inden for kort tid vi siger disse tre components er det ikke bedre i teksten før billedet at henvise til det billede og så efter billedet gå igang med direkte at snakke om model komponenten (Evt se tidligere kommentar er skrevet ind i forslaget)
The model component's responsibility is to implement the logic of the data domains \cite{mvcarticle}.
In other words it is here the main algorithms reside that which provide the solutions for the main function of the application.
The responsibility of the view component is to provide an interface for the user and makes it possible for the user to interact with the application.
The controller component's responsibility is to respond to the user's request and gives a means to make the applications underlying algorithms and logics accessible for the user. \cite{mvcarticle}

The MVC architecture is thus a three-layered structure where each component is loosely dependent on each other.
This gives the developers a possibility of developing each component simultaneously.
This separation between the components also makes it easier to test the application in the test-driven development approach. \cite{mvcarticle}
%Anna: igen det er ikke en test-driven development vi har brugt, men passer stadig fint det med at test tænker jeg?

%The data managed by the application is encapsulated into the model classes.
%These can then be accessed or subscribed to by the views, which the user sees.
%The user then subsequently interacts with the controller, which in turn updates the model.
%The seperation of concerns allows for a quite flexible architecture, where you can change the view without touching the model or controller, or the controller without changing the view and model.\cite{gangoffour}

\subsection{Entity Framework}\label{sec:efcore}

Entity Framework is a framework for mapping object-oriented data structures into a relational database, also referred to as an object-relational mapping framework, for .NET.\cite{efcore}
The job of the Entity Framework is translating the objects in memory into SQL statements so that these objects can be stored in a database.
%Anna, kunne det give mening at skrive SQL ud og så give forkortelsen bagefter? (har ikke nogen stærk følelse for det)
\begin{figure}[H]
	\centering
	\begin{tikzpicture}[node distance=1cm, minimum height=0.6cm]
		\node[process] (net) {ASP.NET Core};
		\node[process] (efcore) [below=0.5cm of net] {Entity Framework Core};
		\node[process] (sql) [below=0.5cm of efcore] {Relational Database};
		\draw[arrow] (net) -- (efcore);
		\draw[arrow] (efcore) -- (net);
		\draw[arrow] (sql) -- (efcore);
		\draw[arrow] (efcore) -- (sql);
	\end{tikzpicture}
	\caption{The relationship between ASP.NET Core and the Relational Database, with the Entity Framework acting as a glue between the two layers}\label{fig:ASP-Entity}
\end{figure}

To use the Entity Framework, everything needed to do in the code is to provide a context object.
which mediates the connection, giving the configuration arguments a connection to a database requires.
This context defines the sets of objects to be stored in the database, and also the database to be used.
The Entity Framework supports multiple kinds of relational databases, from MariaDB to PostgreSQL.
The support for multiple database systems are provided by the respective database providers.
%Anna: er ikke sikker på den sidste sætning er relevant

\subsection{Web Frontends} \label{sec:webfrontends}
The languages of UI (User Interface) on the web are HTML, CSS, and Javascript, and each of these three languages have their own respective roles.\cite{nixonweb}
%Anna: er UI blevet skrevet ud på noget tidpunkt?
%Anna: er det virkeligt de eneste prog eller bare de primære?

\subsubsection*{HTML}

HTML (HyperText Markup Language) is the most important of the three.
It's a markup language that is used to describe the structure and much of the contents of a webpage.
%Anna: Evt beskriv kort hvad et markup sprog er?
\begin{lstlisting}[language=HTML,caption={\color{red}indsæt caption tekst},label=lst:HTML]
<!DOCTYPE html>
<html>
<head>
<title>HTML Example</title>
</head>
<body>
<h1>HTML</h1>
<p>This is a <b>webpage<b></p>
</body>
</html>
\end{lstlisting}
The different tags can contain child tags, which in turn can have their own tags.
This gives the webpage a tree structure.

Two important types of tags are the \texttt{<script>} and \texttt{<style>} tags.
These allow for embedding CSS and Javascript into the page.
The CSS allows for different kinds of styling, such as colouring and positioning of tags.
%Anna: Den sidste sltning høre den ikke til i afsnittet nedenfor?

\subsubsection*{CSS}

CSS (Cascading Style Sheets) allows for defining the style of the document.
%Anna: Document eller webpage?
\begin{lstlisting}[caption={\color{red}indsæt caption tekst},label=lst:CSS]
body {
	background-color: green;
}

p {
	text-color: blue;
}
\end{lstlisting}
As an example the stylesheet in \cref{lst:CSS} colours the background green, and the text in paragraph (\texttt{<p>}) tags blue.
The possibilities with CSS are of course far greater than just colours, as the language also allows for positioning elements, changing their visibility, and far more features than can be covered in this report.\cite{nixonweb}
In working with CSS, the primary focus will be working from an existing CSS Framework called Bootstrap.
Having Bootstrap, which predefines a lot of web components, greatly eases the development of web applications.
While this naturally comes with a trade-off for customization, it was concluded that this project did not need any cutting edge UI to achieve its goals.
%Anna: Evt henvis til hvor det konkluderes eller kort bare skrive hvorfor, hvis vi kan ellers er det ligegyldigt.

\subsubsection*{JavaScript}
%\textbf{JavaScript}

JavaScript is a scripting language built for the web.
%Anna: igen kort beskrive hvad et scripting language er?
JavaScript allows for running code on the page, which can interact with the contents of the page, play sounds, send requests in the background, and much more.\cite{nixonweb}

%Anna er der en grund til vi ikke også giver et emsempel her ligesom med CSS og HTML

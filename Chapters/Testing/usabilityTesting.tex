\documentclass[../../master.tex]{subfiles}
\begin{document}
\section{Usability Testing} \label{sec:usabilitytesting}
%Anja: Mangler generelt en kilde til afsnittet

To gain a better understanding and detecting what issues could possibly arise when Ipsen interacts with the application, usability tests were conducted during the project.
Usability testing results in possible usability issues which are guidelines to further improve the system's design.

\subsection{Usability Test Setup and Structure}
Before usability testing is conducted it is important to find out the purpose and objectives of the test.
A task list will be defined based on the objectives of the usability test that the test subject need to complete using the application.
The observations of the test consists of a test monitor, a screen video recording including audio and a video recording of the view of the PC screen and the test subject's interaction with the keyboard and the mouse.
During the usability testing, the test subject was asked to think aloud while performing a task to easily identify and categorize possible issues in \textit{critical}, \textit{serious} and \textit{cosmetic} which are described below \cite{brugervenligtwebdesign}.

\begin{itemize}
  	\item
  	Critical: An issue is classified as critical if the test subject are unable to continue and stopped completing tasks.
	Critical issues can also involve high irritation while interacting with the application or if the test subject misunderstood one or more significant points of the application.
  	\item
  	Serious: If the test subject is significantly slowed down to completing tasks.
	This means the test subject has to use several minutes on a task.
	\item
	Cosmetic: When the test subject spends few minutes on completing a task, but the issue still needs to be solved.
\end{itemize}

After a usability test has been conducted, a debriefing session immediately begins.
Under the debriefing session, post-test questions will be asked to gain a deeper understanding of the test subject's overall experience with the application.
Here the test subject is able to mention criticism of the interaction, design and suggest solutions and design ideas.
However, a debriefing session's time frame should also allow for discussion if needed.

Evaluation of the test will proceed as follows; while performing the usability test, the test monitor will be taking notes of the possible issues and difficulties that test subject seems to have encountered during interacting with the application.
To better identify and categorize issues, a screen recording with audio is reviewed to ensure there is no missed observations.

\subsection{First Usability Test}\label{firsttest}
Based on the first interview with Ipsen, a prototype for administrator role was created to generate a basic interface design from the requested requirements.
This high-fidelity prototype consisted mostly of HTML and CSS code, see \cref{sec:webfrontends}, with focus on the design without implementation of the functionalities, see \cref{sec:1prototype}.

\subsubsection*{Purpose and Objectives of the First Test}
This usability test was an \textit{exploratory} test type.
The purpose was to represent a rough idea of how the workflow of the different use cases could be operated in the application.
Ipsen would be represented for the interfaces, where she would be able to navigate through the application.
This would provide Ipsen an overview of how the potential layout design might work with the functionalities that later would be implemented.

\subsubsection*{Result of the First Test}
Ipsen's reaction was mostly positive.
She validated the idea of how the department management would be implemented.
Ipsen also liked the overview of the user that both show those who are and those who are not associated to a specific department.
This would be useful if Ipsen forget to add some users to a department.
The same applied to edit associated documents to the department management page.

The "Upload New Version" page and its main idea was that Ipsen would be able to mark new versions as a minor change or major change to notify readers.
If the change was minor, then readers could wait to read the new version.
And if it was a major change, they would need to read it right away.
Ipsen invalidated this functionality, since there was no need for that.

Ipsen was satisfied with how accessing the archive and managing users worked, as it was simple and straightforward.

Ipsen thought that it would be helpful to be able to search for a word in the documents.
For instance, if a job title had change, then she could search for documents that mentioned the job title, so she could easily modify them.
Ipsen wished that this search functionality only applied on active documents.

Whether the application is in English or Danish is according to Ipsen unimportant.
The application would therefore be in English as the report is in English as well.

Ipsen was overall satisfied with the prototype except the upload new version page that needed to be redesigned.
Afterwards the prototype worked as a starting point for the implementation of the application.

\subsection{Second Usability Test}\label{secondtest}
This section will describe the second usability test of type \textit{assessment} test which was conducted on Ipsen and its results. Selected screenshots from this can be found in \cref{sec:2prototype}.

\subsubsection*{Purpose and Objectives of the Second Test}
The majority requested functionalities for the administrator role was implemented in the application at this point.
Therefore this usability test only tested the administrator role part of the application.
The interaction between the test subject and the application must be validated to determine usefulness of the implemented functionalities.
The purpose of the test is to identify general issues concerning the perception of the handbook application for further improvements of the application design.
The research questions, the method description and the task list can be found in \cref{sec:utest2tasklist}.
The tested use cases are listed below.

\begin{itemize}
	\item Access current handbook
	\item Add new file to the system
	\item Approve new version or document
	\item Access archive
	\item Add user
	\item Manage departments
\end{itemize}

The functions of the system that was tested are listed below.

\begin{itemize}
	\item Add document
	\item Add version
	\item Approve version
	\item Request approval
	\item Add user
	\item Add department
	\item Add user to department
	\item Add document to department
\end{itemize}

\subsubsection*{Results}
The identified issues discovered during the usability test is categorized and presented below in \cref{tab:utest2} with explanations following.

\begin{table}[H]
	\begin{center}
	\begin{tabular}{| c | m{21em} | c | c | c | c | c |}
		\hline
		\# & \textbf{Identified usability issues} & Critical  & Serious & Cosmetic \\
		\hline
		 1 & Confusion regarding add and upload a new document   & x &  &  \\
		\hline
		 2 & No overview of existing document ID when creating a new document &  & x & \\
		\hline
		 3 & Confusion when creating a new version to a document & x & &  \\
		\hline
		4 & No indication when a version is sent to approval & x & & \\
		\hline
		5 & No indication on which page is currently showing &  &  & x \\
		\hline
		6 & Negative document ID &  & x & \\
		\hline
		7 & Not able to correct/modify a document ID & & x &  \\
		\hline
		8 & Department's associated users only presents as email &  &  & x \\
		\hline
	\end{tabular}
	\end{center}
	\caption{Identified issues for the second usability test}\label{tab:utest2}
\end{table}
The identified issue '1' occurred when Ipsen tried to create a new document and add a version to the document.
Ipsen thought that it is possible to both add a document and upload at the same time.
The application at the current state would work, such that, a user had to add a document first by entering the document name, chapter ID and section ID.
Afterwards, the user had to find the document in the handbook and upload a new version.
This created confusion, which caused an interrupted work flow with the handbook.

The identified issue '2' took place when Ipsen attempted to create a new document.
Ipsen had to go back to the handbook's overview to check the existing document ID to be able to find out which chapter the new document should belong to.

The identified issue '3' occurred when Ipsen tried to create a new version to an existing document.
It was unclear where to upload a new version.
Therefore, Ipsen tried to add a new document and then uploaded a new version.
The upload button was not noticeable and it took time before Ipsen found it.

The identified issue '4' turned up as Ipsen tried to send a version to approval and then had to approve the version herself.
Ipsen did not detect it when she had sent the version to approval or whether she had just approved the version.
The issue could be solved by the system providing feedback on the action.

The fifth identified issue is the confusion that occurred while Ipsen attempted to approve a version while in the wrong view.
This could be because the sidebar did not indicate which page she was currently on.

The identified issues '6' and '7' appeared, when Ipsen made a mistake, as the application allowed her to enter a negative chapter and section ID.
Ipsen felt the need to correct the mistake, which the application did not support at that moment.
Ipsen wished to be able to delete a document, that did not contain a version yet, if a mistake occurred.

The identified issue '8' occurred as Ipsen assigned a user to a department and noticed that users were presented as an email address.
Ipsen preferred to have a name displayed instead of an email address.

In the debriefing session Ipsen mentioned that she was surprised that the application required user to attached a PDF file as she has seen other document version control system has their own file editor.
However, Ipsen could see the reason behind it and commented that it is a good approach.
When informed, she also recalled specifically asking for this approach.

Ipsen also remarked that the features' names were clear and understandable.

During the test session it had been clarified that the administrator must be able to use his right to get an awaiting approval approved without the other approvers accepting it.
The situations where other approvers either do not get it done, are too slow or are on a vacation could occur.
The administrator right will therefore prevent possible bottleneck in the work flow.

\subsection{Third Usability Test}\label{thirdtest}
The third usability test was conducted on Ipsen right after the second test had finished.

\subsubsection*{Purpose and Objectives of the Third Test}
The second usability test only had focus on the administrator role.
The purpose of this third test was to validate the idea of how the interfaces for a writer and reader role could possibly look like.

\subsubsection*{Results of the Third Test}
The major difference of the interface between administrator role from writer and reader role was the sidebar on the left, see \cref{fig:mockupSidebar}.
Since writer and reader would not be able to access the most of the functionalities available on the sidebar.
The sidebar would therefore only appear for administrator.
Ipsen agreed on the idea to keep the interface as simple as possible for both writers and readers.

\begin{figure}[H]
	\centering
	\begin{subfigure}[b]{0.48\textwidth}
		\includegraphics[width=\textwidth]{billeder/ForsideAdmin.jpg}
		\caption{Mockup main interface for administrator}
	\end{subfigure}
	\quad
	\begin{subfigure}[b]{0.48\textwidth}
		\includegraphics[width=\textwidth]{billeder/ForsideWriterReader.jpg}
		\caption{Mockup main interface for writer and reader}
	\end{subfigure}
	\caption{Mockup main interfaces comparision}\label{fig:mockupSidebar}
\end{figure}

In the mockup the handbook page consisted of division of relevant document that would be showed first on the top and the full handbook underneath.
For the reader and writer role the relevant document would consist of those associated documents of the department they belonged to.

The idea of implementing a preview, when user clicked on the document name, was also showed to Ipsen.
The user would have an opportunity to see a preview of a document before chosen to display it full screen, see \cref{fig:mockupPreview}
This enable the user to better navigate through and still have an overview of the handbook which Ipsen also agreed upon.

\begin{figure}[H]
	\centering
	\begin{subfigure}[b]{0.48\textwidth}
		\includegraphics[width=\textwidth]{billeder/PreviewVersion.jpg}
		\caption{Mockup interface of document preview}
	\end{subfigure}
	\quad
	\begin{subfigure}[b]{0.48\textwidth}
		\includegraphics[width=\textwidth]{billeder/FullView.jpg}
		\caption{Mockup interface of document full view}
	\end{subfigure}
	\caption{Mockup interfaces of document preview and full view}\label{fig:mockupPreview}
\end{figure}

An interface change of the sidebar was also represented which applied to every role.
Beforehand, the sidebar only contained informative text which in the mockup now had added suitable icons as well.
The user had two options with the sidebar.
Either, the sidebar could be folded to only consist of icons, see \cref{fig:mockupSidebarIcon}, or unfold to see both the icons and informative text.
This new sidebar design was validated as well by Ipsen.

\begin{figure}[H]
	\centering
		\includegraphics[width=0.7\textwidth]{billeder/ForsideFoldedSidebar.jpg}
	\caption{Mockup interface of folded sidebar with icons}\label{fig:mockupSidebarIcon}
\end{figure}

The last implementation of the application could continue to prepare for the last usability test of type validation test.

\subsection{Fourth Usability Test}\label{fourthtest}
The fourth usability test was conducted on Ipsen, Ipsen's husband and their daughter.
The reader, writer and administrator role were all tested in the application this time around.

\subsubsection*{Purpose and Objectives of the Fourth Test}
The usability issues from the first test had been corrected and ready to be tested to determine whether the new functionalities are improved or not.
All the roles needed to be tested, as well as the recently added functionality, which are required by the system definition.
The department and user management in the application will also be a part of the testing.
The task list for the fourth usability test can be found in \cref{sec:utest4tasklist}.
The use cases that will be tested are listed below.

\begin{itemize}
	\item Access current handbook (exclude the first time login)
	\item Add new file to the system
	\item Approve new version or document
	\item Access archive
	\item Add user
	\item Edit user information
	\item Manage users
	\item View who has read a document
	\item Manage departments
\end{itemize}

The functions that will be tested can be seen below.

\begin{itemize}
	\item Add document
	\item Add version
	\item Approve version
	\item Request approval
	\item Update user
	\item Delete user
	\item Add user
	\item Add department
	\item Add user to department
	\item Add document to department
	\item Download archive
\end{itemize}

\subsubsection*{Results of the Fourth Test}
The detected usability issues are present and categorize in table
\Cref{tab:utest4} which will be clarified afterwards.

The preparation for the fourth usability test was not done thoroughly because of the time pressure and poor planning.
The documents in the handbook, that the test subjects had to find, according to the described task list, were not the same in the application.
The application had not been tested thoroughly to check whether the test subject would be able to complete the tasks.
Therefore, lots of errors appeared during the tests.

\begin{table}[H]
	\begin{center}
	\begin{tabular}{| c | m{15em} | c | c | c | c | c | c |}
		\hline
		\# & \textbf{Identified usability issues} & Role & Critical & Serious & Cosmetic \\
		\hline
		 1 & The notification button are not visible & All & & & x \\
		\hline
		 2 & Confusion concerning the working file & A, W & x & & \\
		\hline
		 3 & Confusion how to add more than one approver & A, W & x & & \\
		\hline
		4 & No document ID when adding associate documents to a department & A & & & x\\
		\hline
		5 &  Confusion about upload and download icons & A, W & x &  &\\
		\hline
		6 & No approve button on awaiting approval document's view & A, W &  & & x\\
		\hline
		7 & No cancel button when edit department & A & & & x\\
		\hline
		8 & Deleted user should not appear while edit department & A & & & x \\
		\hline
		9 & Edit department user list present username & A &  & & x \\
		\hline
		10 & The cursor needs be a pointer when hovering over a row & A &  &  & x \\
		\hline
		11 & When deleting a user the dialog box  does not specificity which user & A & & & x\\
		\hline
		12 & No indication of who are currently logged in & All & & & x \\
		\hline
		13 & No indication that the awaiting approval had been approved by you & A, W & & x & \\
		\hline
	\end{tabular}
	\end{center}
	 {\raggedright Remark: A denotes administrator role, W denotes writer role and R denotes reader role.\par}
	\caption{Identified issues for the fourth usability test}\label{tab:utest4}
\end{table}

The identified issue '1' occurred when Ipsen was asked to find the notification button.
However, there was no notification alert as the notification functionalities was not completed at this point.
Ipsen commented that the notification button was quite invisible, as the notification icon had the almost the same color as the navigation bar.

The identified issue '2' appeared when Ipsen attempted to upload a new version to the handbook.
New functionalities had been added, which was the option to both upload PDF file and a working file (excel, word etc.).
Administrator and writer would be able to download working file and easily edit the file.
Ipsen did not understand what was meant by uploading a working file and its purpose.
A conclusion had been reached to only upload PDF file, for the sake of simplicity.
Since administrator would usually store all the working files on the pc regardless.

\begin{figure}[H]
	\centering
		\includegraphics[width=0.7\textwidth]{billeder/WorkingFile.png}
	\caption{Upload new version page with working file option}\label{fig:WorkingFile}
\end{figure}

The identified issue '3' was discovered when there was confusion of how to add multiple approvers while uploading a new version.
To solve this issue a guide text will be added.

The identified issue '4' was noticed while adding documents to a department and the document ID was missing.
Both document ID and name needed to be listed.

The identified issue '5' occurred when asking to upload a new version to a document.
The upload and download icons were not intuitive enough on their own, on the main page, and they also looked similar, see \cref{fig:MainInterfaceAdmin}.
Ipsen preferred a button with text instead, as no doubts would happen.
Ipsen also mentioned how the upload icon on the main interface would not reflect the right mindset in the workflow.
An "Add" icon or button would be more suitable, where afterwards a user would be able to press on an "upload" icon or button.

\begin{figure}[H]
	\centering
		\includegraphics[width=0.7\textwidth]{billeder/MainInterfaceAdmin.png}
	\caption{Main interface for admin role}\label{fig:MainInterfaceAdmin}
\end{figure}

The identified issue '6' was discovered when Ipsen needed to approve an approval request.
When entering the document view a approve button should also appear so administrator did not need to go back to the previous page to press the approve button.

The identified issue '7' was noticed when adding user and document to a department where no cancel button appeared.

The identified issue '8' occurred when adding user to a department and deleted user still presented in the list, as a long random character and number combination.
The identified issue '9' was also discovered, that user represented as username and not name.

The identified issue '10'  and '11' was discovered when deleting a user.
An error, such as accidentally deleting an unintended user could most likely occur.
Ways to indicate and make it clear are as follow; the user row that the cursor is hovering over will be highlighted or dialog box needs to specify which specific user will be deleted.

The identified issue '12' occurred when Ipsen's husband mentioned that it would be helpful, if the interface showed which user that was currently logged in the application.

The identified issue '13' was discovered once administrator had approved an awaiting approval request that still need to be approved by others.
No indication was showed that it already had been reviewed and approved.
The user must therefore remember which document they had approved.


During the debriefing session Ipsen was asked if there was other functionalities she desired, that the application could provide.
She mentioned that she wished to be able to access a list of departments a user was associated to, and a list of all the documents a user did not yet read.
Administrator could make a mistake and add a user to a wrong department for that reason.
Ipsen also suggested, that if a user did not read an associated document for over a week, that the department head would be alerted.
If a user still had not read by over a period of two weeks, that the administrator would be alerted.

It had also come to the conclusion, that the department head should be the one who keeps track and ensures that the employees had read the associated documents.

Furthermore, Ipsen had hoped that the application would be able to change the date and the version number in the actual document header.
However, even though the application does keep track of the date and the version, readers also need to know whether they stand with the valid version in printed form, for more information see \cref{sec:4notice}.

The writer rights, regarding if a writer are allowed to archive a document in the handbook, had also been clarified.
Ipsen pointed out that only the administrator is able to archive documents.
\end{document}

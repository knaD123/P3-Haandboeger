\chapter{Interview} \label{sec:interview}

There were several aspects to consider in advance of holding an interview.
First is the type of interview to hold, of which there are \textit{unstructured interviews, structured interviews}, and \textit{semi-structured interviews} \citep{interactionhci}.
Each of these types have their pros and cons: A structured interview generally results in shorter and more precise answers, whereas an unstructured interview allows the conversation to be more spontaneous and results in more unexpected answers and more rich information \citep{interview}.
The semi-structured interview was usually chosen as this is a middle ground between the two interview types in the spectrum.
This both allows for the interviewer to ask a range of questions, but gives enough leeway to ask follow-up questions outside of the prepared questions depending on the interviewee’s answers.
Though the other structures were also used once or twice.

In preparation for most interviews an interview guide was made.
An interview table was formed where research questions where formulated in the left-hand side and interview questions in the right-hand side.
Here the research questions were academically formulated to give an overview of what kind of information the interviewer seeks, and the interview question is written in everyday language and seeks to answer the research question \citep{interview}.
Structuring the interview guide this way makes it more manageable to see what kind of information we seek to acquire before we formulate the corresponding questions.
It should be noted that it is only for the group members benefit to structure the interview guide this way, and that the interviewee doesn’t necessarily gets to see the guide or the questions.